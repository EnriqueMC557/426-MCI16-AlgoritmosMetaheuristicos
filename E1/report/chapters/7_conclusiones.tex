Dado nuestro conocimiento previo, sabemos que el mínimo de la función de Rosenbrock se encuentra en $(x_1, x_2) = (1, 1)$, donde el valor de la función es $f(x_1, x_2) = 0$. Tomando esto como referencia, podemos considerar los valores de aptitud obtenidos por el algoritmo como una métrica de error. En este caso, hemos obtenido un error promedio de $0.1791$, lo cual es un valor bastante aceptable.

A pesar de que el algoritmo no logró encontrar el mínimo exacto de la función, logró acercarse significativamente. La limitación que enfrentamos podría estar relacionada con la elección de la codificación utilizada, en este caso, una codificación octal. Es posible que el valor preciso del mínimo no se encuentre dentro de los valores posibles para la codificación, o que la precisión generada por esta codificación ocasione que el algoritmo oscile en valores muy cercanos al mínimo real.

En conclusión, hemos demostrado la utilidad de los algoritmos genéticos para determinar mínimos en una función, incluso en un caso de dos variables como la función de Rosenbrock. Esto abre la puerta para utilizar estas técnicas heurísticas en la búsqueda de mínimos o máximos en una amplia variedad de funciones, lo que puede tener aplicaciones prácticas en la resolución de problemas del mundo real. Aunque no hemos alcanzado el mínimo exacto en este caso, hemos demostrado que estos algoritmos pueden ser valiosos en la optimización de funciones complejas.