El objetivo principal de este trabajo, encontrar el precio optimo de los productos lacteos, se consiguió de forma satisfactoria. Es cierto que este ejercicio se realizó en gran parte con datos sintéticos, pero la implementación de este se deja abierta para el uso de datos reales del mercado y la industria específica.

Nuevamente se ha observado como los algoritmos genéticos son una herramienta muy valiosa para optimizar procesos en tiempos muy cortos, donde si utilizamos una codificación y función de aptitud adecuada podemos llegar a resultados óptimos.

Para el caso específico del análisis de datos, se puede destacar que siempre es necesario realizar análisis de diversas formas, ya que un análisis básico no siempre nos puede dar la información correcta. En este caso el primer análisis nos indicaba que la variación de precio era mínima, pero al corroborar con análisis más detallados se observó que en realidad no existe un precio que permanezca constante a lo largo del tiempo.

