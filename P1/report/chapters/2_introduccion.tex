Los algoritmos genéticos son una técnica de inteligencia artificial que se inspira en la evolución biológica y la genética para resolver problemas complejos. Estos algoritmos se basan en la idea de que el individuo más adaptado al medio es el que sobrevive y transmite sus características a sus descendientes. Así, los algoritmos genéticos simulan el proceso de selección natural mediante la manipulación de cadenas de símbolos que representan posibles soluciones a un problema dado.

Los algoritmos genéticos han tenido un impacto significativo en el campo de la computación y la inteligencia artificial. Su enfoque de búsqueda basado en la evolución ha demostrado ser efectivo para resolver problemas complejos en los que los métodos tradicionales pueden ser ineficientes o inadecuados. Los algoritmos genéticos han inspirado muchas otras técnicas de optimización y metaheurísticas, lo que ha llevado al desarrollo de una amplia gama de algoritmos evolutivos.

Algunas de sus aplicaciones notables incluyen:

\begin{itemize}
	\item Diseño de Circuitos Electrónicos.
	\item Optimización de Procesos Industriales.
	\item Diseño de Redes y Enrutamiento.
	\item Diseño de Estructuras.
	\item Aprendizaje Automático.
\end{itemize}
