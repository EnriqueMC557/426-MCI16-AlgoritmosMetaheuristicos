En términos generales, cada uno de los algoritmos exhibe resultados alentadores en relación tanto a la aptitud promedio como a la mejor aptitud obtenida. No obstante, la elección del algoritmo óptimo está totalmente ligado a las prioridades del problema en cuestión. Si la velocidad de convergencia es una consideración primordial, el Algoritmo 2 podría destacarse, dado su logro de una excelente mejor aptitud en tan solo dos generaciones. En caso de que tanto la aptitud promedio como el tiempo necesario para converger sean factores determinantes, valdría la pena contemplar la viabilidad de los Algoritmos 1 o 3. Por otro lado, el Algoritmo 4, a pesar de su rápida convergencia, no alcanza la deseada mejor aptitud, sugiriendo así la posibilidad de ajustar su diseño.

Por último, es relevante mencionar que si bien todos los algoritmos han logrado la convergencia esperada sin mostrar signos de estagnación, la elección adecuada de los criterios de paro resulta crucial. En el caso específico de esta práctica, la elección de un épsilon más elevado con seguridad hubiera generado una mejora en la aptitud de todos los algoritmos.