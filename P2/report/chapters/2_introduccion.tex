El Problema del Vendedor Viajero (TSP, por sus siglas en inglés: Traveling Salesman Problem) es uno de los problemas de optimización combinatoria más estudiados y conocidos en el ámbito de las ciencias de la computación y la investigación operativa. El desafío radica en encontrar el recorrido más corto que permita visitar una serie de ciudades y regresar al punto de origen, sin pasar más de una vez por cada ciudad. A pesar de su enunciado sencillo, el TSP es un problema NP-difícil, lo que significa que no se conoce una solución exacta que funcione en tiempo polinómico para instancias arbitrariamente grandes.

Dado que el número de posibles recorridos aumenta de manera factorial con el número de ciudades, resolver el TSP de forma exacta para una gran cantidad de ciudades se convierte en una tarea computacionalmente prohibitiva. Por lo tanto, la búsqueda de soluciones aproximadas y heurísticas se vuelve esencial.

En este contexto, los Algoritmos Genéticos (AG) han emergido como una herramienta poderosa y versátil para abordar el TSP. Inspirados en la teoría de la evolución natural, los AG simulan el proceso de selección natural donde los individuos más aptos son seleccionados para la reproducción, con el objetivo de producir descendencia de alta calidad para la próxima generación. En el caso del TSP, un individuo puede representar una posible solución o recorrido, y su aptitud puede estar relacionada con la distancia total o coste de ese recorrido.

La naturaleza adaptativa y probabilística de los AG los hace especialmente adecuados para explorar el amplio espacio de soluciones del TSP. Mediante la combinación de soluciones existentes (cruza) y la introducción de pequeñas perturbaciones aleatorias (mutaciones), los AG pueden explorar y explotar diferentes regiones del espacio de búsqueda, convergiendo a menudo a soluciones cercanas al óptimo global.
