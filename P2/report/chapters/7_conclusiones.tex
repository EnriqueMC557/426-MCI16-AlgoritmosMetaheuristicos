Al analizar los resultados de todos los algoritmos, se destaca el rendimiento superior del algoritmo número 4. A primera vista, uno podría atribuir este éxito a la combinación efectiva de los mecanismos de selección por torneo, Cx y la mutación inversa. Sin embargo, es plausible que la verdadera razón detrás de estos resultados sea la cantidad de individuos utilizados en este algoritmo: 1,000 individuos. Esta amplia muestra permitió al algoritmo explorar un espacio de búsqueda más extenso en comparación con los demás, lo que si bien implicó un mayor tiempo de ejecución, también se tradujo en resultados superiores.

Aunque los demás algoritmos no tuvieron un rendimiento deficiente, y partiendo de la premisa de que este problema no tiene una única solución sino un conjunto de soluciones óptimas, se podría inferir que cualquier algoritmo genético es adecuado para resolver el TSP. No obstante, es importante considerar que una población más numerosa tiende a garantizar resultados más óptimos.
