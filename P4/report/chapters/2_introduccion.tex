Sea una planta de producción continua de papel. La misma consiste de un mesa formadora en donde se pone la pasta de papel (fibras celulósicas más otros componentes). La mesa formadora alimenta una serie de rodillos calentados a alta temperatura que secan y comprimen la pasta, generando, de esa manera, el papel. Se pueden producir diferentes colores y calidades de papel. Para cada par de estos corresponden capacidades máximas de producción. (El papel se mide por el gramaje que es el peso en gramos de un metro cuadrado del mismo). Cuanto más alto es el gramaje, menos es la producción horaria, debido a las limitaciones impuestas por la máxima capacidad de trabajo de la máquina. Cuanto más gramaje mayor es el flujo de masa y por lo tanto mayor es la necesidad de potencia para generar el suficiente calor para secarlo y la suficiente presión para laminarlo y llevarlo al espesor deseado. A menores gramajes (papel más fino) la limitación está dada por la resistencia a la rotura del papel, ya que el aumento de la capacidad de producción requeriría velocidades que harían que el papel se rompiera obligando a parar la planta y reiniciar el proceso de producción.