La implementación exitosa de las islas, con la isla maestra destacando en su avance hacia una aptitud superior, sugiere una eficaz aplicación de algoritmos genéticos. En la fase de experimentación, la viabilidad de aumentar el número de islas para explorar un amplio espacio de búsqueda frente al tiempo de ejecución se plantea como un siguiente paso. Además, la implementación actual con DataFrames (pandas), aunque permite la visualización, podría optimizarse mediante estructuras de datos más ligeras como listas o diccionarios para mejorar la eficiencia del programa. Esta optimización es crucial, especialmente en aplicaciones industriales, donde los algoritmos genéticos desempeñan un papel fundamental en la optimización de procesos complejos, la toma de decisiones multicriterio y el diseño de sistemas robustos frente a la variabilidad y la incertidumbre.